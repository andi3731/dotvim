\documentclass{beamer}
\usepackage[latin1]{inputenc}
\usepackage[T1]{fontenc}
\usepackage{lmodern}
%\usepackage{times}
%\usepackage{kpfonts}
%\usepackage{pxfonts}
%\usepackage{txfontsb}

\setbeamerfont{structure}{family=\rmfamily,shape=\itshape} 
%\usefonttheme{serif} 
\usefonttheme[stillsansseriftext,stillsansserifmath]{serif} 




\usepackage{amssymb,amsmath, amsthm}
\usepackage[]{color}
\usepackage{graphicx}
\usepackage{booktabs}
\usepackage{subfig} 

%\newtheorem{theorem}{Theorem}
%\newtheorem{result}{Result}

\def\Rreal{\mathbb{R}}
\def\Zint{\mathbb{Z}}
\newcommand{\abs}[1]{\lvert #1 \rvert}
\newcommand{\norm}[1]{\lVert #1 \rVert}
\newcommand{\normone}[1]{{\lVert #1 \rVert}_{1}}
\newcommand{\norminf}[1]{{\lVert #1 \rVert}_{\infty}}
\newcommand{\normtwo}[1]{{\lVert #1 \rVert}_{2}}

\definecolor{MyDarkBlue}{rgb}{0.1,0,0.55}
\newcommand{\todolater}[1]
{\begin{center} {\color{red}
  \textbf{ {\#\#\#\hfill\#\#\#}\\{#1}\\\#\#\#\hfill\#\#\#}} \end{center} } % todolater
\newenvironment{modifa}
{\color{MyDarkBlue}}
{}
\newenvironment{todel}
{\color{red}}
{}

\newcommand{\todo}[1]
{\begin{tiny}
  \begin{center} {\color{red}
  \textbf{ {\#\#\#\hfill\#\#\#}\\{#1}\\\#\#\#\hfill\#\#\#}}
\end{center}
\end{tiny}} % todolater

\usetheme{Warsaw}
%\usetheme{Boadilla}
%\usetheme{Darmstadt}
%\usetheme{Dresden}


\title{Convex Relaxations for Cubic Polynomials Problems}
\author[In\'acio, Ahmed, Realff]{Helder In\'acio \\ Shabbir Ahmed \\ Matthew Realff}
\date{November 10, 2010}
\institute[GATech]{Georgia Institute of Technology}
\begin{document}

\begin{frame}
\titlepage
\end{frame}



\begin{frame}{Introduction}

  We consider a problem in the form : 
  \begin{displaymath}
    \begin{array}[h]{rcl}
      \min\limits_x & c'x & \\
      \text{subject to } & g_s(x) & \leq 0 , \quad s \in S\\ 
      & \multicolumn{2}{l}{ 0 \leq x^l \leq x \leq x^u < \infty} \\
    \end{array}
  \end{displaymath}

where $g(x)$ is a polynomial function of degree up to 3:
\begin{displaymath}
  g_s(x) = \sum_{i \in I} a_i x^i, \quad s \in S
\end{displaymath}
with $a_i$ a real coefficient and 
%$x^i = x_{i_1}^{i_1}\cdot x_{i_2}^{i_2}\cdots x_{i_n}^{i_n}$, $i_1, i_2, \dots, i_n \in
%\mathbf{N}$ and $\sum_k i_k \leq 3$.

$x^i = \prod_{k=1}^{n} x_k^{i_k}$, $i_k \in \mathbf{N_0}$ and $\sum_{k=1}^n i_k \leq 3$.

\end{frame}

\begin{frame}{Introduction}

  \begin{itemize}
    \item 
  If all $g_s(x)$ are convex functions then problem is convex, 
  so any local optimum solution is also global.
  \item
    In general functions $g_s(x)$ are not convex.
  \end{itemize}

  Approach: \\
  For nonconvex $g_s(x)$ replace it by convex approximation.
  \todo{talk about branch and bound}
  
\end{frame}

\begin{frame}{Convex under approximation }

  \todo{ Picture of polyhedral under approximation}
  \todo{ Figure of convex relaxation }

  \todo{ (the convex should in this case be better than polyhedral)}
  
\end{frame}

%\begin{frame}{Introduction}

  %\begin{itemize}
    %\item 
      %Many approaches know and implemented for solving this problem.
      %Commercial software code BARON includes and open source code project COUENNE 
      %find global solutions to this problem.

  %\end{itemize}
%\end{frame}

\begin{frame}{Convex Relaxations}
  We consider for simplicity a single constraint 
  \begin{displaymath}
    g(x) \leq 0
  \end{displaymath}

  We want to build a convex relaxation for the set 
  \begin{displaymath}
    S = \left\{ x \mid g(x) \leq 0 \right\}
  \end{displaymath}
  that is a set $S^C$ such that
  \begin{itemize}
    \item 
      $S^C$ is convex.
    \item
      $S \subseteq S^C $
  \end{itemize}
  Ideally we would like to have that if $S^{C'}$ is another convex set such that 
  $S \subseteq S^{C'}$ then $S^{C} \subseteq S^{C'}$, that is $S^C$ is the convex hull of
  $S$.
\end{frame}

\begin{frame}{Convex Relaxations}
  
  Unfortunately this approach not attainable in general.
  \todo{  (Mention Sahinidis and Bao here?) }

  What can be done? \\

  build relaxations for individual terms for which usually a tight convex relaxation is
  know.

\end{frame}

\begin{frame}{Quadratic Case}
  We consider for simplicity a single constraint 
  \begin{displaymath}
    g(x) \leq 0
  \end{displaymath}

  If there are no terms of order 3, then we can write

  \begin{displaymath}
    g(x) = a_0 + \sum_{i \neq j} b_{ij} x_i x_j + \sum_{i=1}^n c_i x_i^2
  \end{displaymath}
\end{frame}

\begin{frame}{Quadratic Case}
 In this case we have to deal with terms of the form

 $ \alpha x \cdot y $ and $\beta x^2$. \\
 Consider $x \cdot y$ with $x \in [x^l, x^u]$ and $y \in [y^l, y^u]$.
 To build a relaxation of this term we introduce a new variable $z$ and the two linear
 constraints
 \begin{displaymath}
   \begin{array}[h]{rl}
     z \geq & y^l \cdot x + x^l \cdot y - x^l \cdot y^l \\
     z \geq & y^u \cdot x + x^u \cdot y - x^u \cdot y^u \\
   \end{array}
 \end{displaymath}

 In the case of a $-x \cdot y$ term the relaxation is
 \begin{displaymath}
   \begin{array}[h]{rl}
     z \leq & y^l \cdot x + x^u \cdot y - x^u \cdot y^l \\
     z \leq & y^u \cdot x + x^l \cdot y - x^l \cdot y^u \\
   \end{array}
 \end{displaymath}
\end{frame}

\begin{frame}{Quadratic Case}
  Previous relaxations are know as McCormic convex and concave envelopes of $x \cdot y$
  in $[x^l, x^u] \times [y^l, y^u]$.
  \todo{picture of relaxations of x y}

  For the terms $x^2$ we have that this term is convex, so we can leave it unchanged a
  possible relaxation.\\
  However if we require the relaxation to be polyhedral we can use:
  \begin{displaymath}
    z \geq (x^i)^2 + 2 x^i \cdot (x-x^i) = -(x^i)^2 + 2 x^i \cdot x, \quad i=1,\dots N_i
  \end{displaymath}
  as an polyhedral approximation.

  We note that each of the previous inequalities is simply a subgradient inequality for
  $x^2$ at a point $x^i$.

  For $-x^2$ the best approximation we can have is the secant approximation
  \todo{formula for secant approximation}
  
\end{frame}

\begin{frame}{Beyond the Quadratic Case}

  \todo{mention factorable programming}

  \todo{mention in some cases might be better direct approach}
  
\end{frame}

\begin{frame}{Convex Relaxations for Cubic Problems}
  
\end{frame}


\end{document}
